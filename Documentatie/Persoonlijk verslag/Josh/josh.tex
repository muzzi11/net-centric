\documentclass[12pt]{article}

\usepackage[dutch]{babel}
\usepackage{a4wide}
\usepackage{amsmath}
\usepackage{enumerate}
\usepackage{url}
\usepackage{graphicx}

\author{Jos Bonsink}

\begin{document}

\title{Persoonlijk verslag - Project Thunder}
\maketitle

\section*{Inleiding}
Dit is mijn persoonlijk verslag van mijn uitgevoerde activiteiten.

\section*{Week 1}
\subsection*{Maandag}
Brainstormen over mogelijke onderwerpen met Mustafa. Uiteindelijk besloten om een 
akoestische lokalisatie systeem met Android telefoons te gaan maken. De wiskunde voor het relatief bepalen van posities hebben we in de middag uitgewerkt.

\subsection*{Dinsdag}
Nadat we de lange termijn planning hadden uitgestippeld, kon ik beginnen met het bluetooth gedeelte. Ik had bluetooth device discovery geïmplementeerd met activities voor weergave. Daarna begon ik aan bluetooth verbindings code.
Ook had ik een knop aangemaakt om bluetooth modus in te schakelen.

\subsection*{Woensdag}
Ik had de discovery activity aangepast, zodanig dat dubbele entries niet meer worden getoond. Discovery wordt nu in een apparte thread uitgevoerd. Ik had ook code geschreven om entries na een bepaalde tijd uit de ArrayAdapter te verwijderen. Na veel moeite gelukt maar deze functie was uiteindelijk niet nodig. Dus de code heb ik verwijderd. Er is namelijk gekozen om de gehele lijst te wissen nadat discovery is afgerond.

\subsection*{Vrijdag}
Verder gewerkt aan het onderling verbinden van telefoons met bluetooth. Veel problemen tegengekomen. Discovery werkte ineens niet meer goed. Het paren van telefoons mislukte telkens. Af en toe werkte het beter door de telefoons opnieuw op te starten. Testen en debuggen duurt erg lang allemaal. Uiteindelijk nog niet helemaal gelukt om 4 telefoons met elkaar te laten connecten. Knop voor server/client switch aangemaakt en later verwijderd. Altijd discovery aangezet, knop voor discovery verwijderd. Altijd connecten met devices met als naam Project Thunder.

\section*{Week 2}
\subsection*{Maandag}
Geprobeerd een oplossing te vinden voor de problemen die ik vrijdag was tegengekomen. Niet gelukt, telefoons weigerde om met meer dan 1 andere telefoon te pairen. Vervolgens een library geprobeerd, dit mislukte ook. Vervolgens de oude code weer hersteld en dingen aangepast. Samen met Mustafa weer vanaf nul begonnen.

\subsection*{Dinsdag}
Werken aan bluetooth implementatie met Mustafa. Het experimenteren met de telefoons. Kijken hoe lang het duurt voordat telefoons met elkaar zijn verbonden. Heel veel problemen gehad. Vaak moeten alle telefoons opnieuw worden opgestart. Er is gebleken dat meer dan 2 connecties in de praktijk niet mogelijk is.

\subsection*{Woensdag}
Mergen van de oude bluetooth implementatie en het opschonen van de nieuwe implementatie. De client/server code in de activity verplaatst naar een apparte class. Vervolgens ge\"implementeerd dat er berichten ontvangen en verstuurd kunnen worden.

\subsection*{Vrijdag}
Begonnen met het automatish aanmaken van een netwerk van android toestellen met behulp van bluetooth. Met een druk op een knop wordt het proces gestart. De telefoon zoekt de eerste beste match en connect daarmee. Vervolgens zoekt die telefoon weer naar een volgende, etc. De mac addressen van de telefoons aanwezig in het netwerk worden in een listview getoond.

\section*{Week 3}
\subsection*{Maandag}
Samen met Mustafa netwerk code verder gestest en bugs geplet. Een broadcast functie ge\"implementeerd. Enkele functies uit de Bluetooth service gehaald en in apparte handlers gestopt. Enkele handlers toegevoegd om onderlinge communicatie mogelijk te maken. Er kan maar een maximaal aantal keer worden gezocht naar andere bluetooth apparaten. Hierna wordt er van gegaan dat het netwerk compleet is. 

\subsection*{Dinsdag}
Bluetooth communicatie verder uitgewerkt. Het is nu mogelijk om berichten terug te sturen naar de afzender. Enkele problemen opgelost die tijdens het testen boven het water zijn komen drijven. 

\subsection*{Woensdag}
De netwerkprotocollen zijn gladgestreken. Het testen van het systeem is eindelijk mogelijk. Enkele experimenten met Mustafa uitgevoerd en het systeem zo goed mogelijk afgesteld. Aan het einde van de dag samen met Mustafa aan de poster gewerkt.

\subsection*{Woensdag}
Poster afgemaakt.

\subsection*{Vrijdag}
Samen met Mustafa de posterpresentatie gehouden. Het was een leuke ervaring.

\end{document}

