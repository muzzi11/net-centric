\documentclass[12pt]{article}

\usepackage[dutch]{babel}
\usepackage{a4wide}
\usepackage{amsmath}
\usepackage{enumerate}
\usepackage{eurosym}
\usepackage{url}
\usepackage{graphicx}

\author{Jos Bonsink \& Mustafa Karaalioglu}

\begin{document}

\title{Notulen - Project Thunder}
\maketitle

\section*{Inleiding}
Elke ochtend bespreken Mustafa en Jos welke problemen ze de vorige dag waren tegengekomen en wat hun planning van de dag is.

\section*{Week 1}
\subsection*{Maandag, 10-06-2013}
Mustafa en Jos bespreken project idee\"en en eventuele studiegenoten waarmee ze een team kunnen vormen.
\subsection*{Dinsdag, 11-06-2013}
Mustafa en Jos bespreken de lange termijnplanning en de taakverdeling. Uiteindelijk is besproken dat Mustafa aan het audio gedeelte zal gaan werken en Jos aan het bluetooth gedeelte. Mustafa zal beginnen met het afspelen van audio en Jos zal beginnen met het vinden van bluetooth apparaten.
\subsection*{Woensdag, 12-06-2013}
Mustafa heeft de vorige dag mogelijk gemaakt dat geluidsbestanden kunnen worden afgespeeld en dat er geluid kan worden opgenomen. Jos heeft een activity gemaakt waarbij gevonden bluetooth apparaten op het scherm worden getoond. Mustafa en Jos weten wat ze moeten doen en gaan weer verder.
\subsection*{Vrijdag, 14-06-2013}
Jos kan eindelijk beginnen met het onderling verbinden van bluetooth apparaten. Mustafa had woensdag problemen met zijn recorder. Mustafa kiest ervoor om een beepgenerator te gaan gebruiken.


\section*{Week 2}
\subsection*{Maandag}
Het onderling verbinden is deels gelukt, er zijn nog wat kleine problemen die opgelost moeten worden. Jos zal zich hier vandaag mee bezig houden. De nieuwe recorder en beep generator zijn ge\"implementeerd, ook wordt er al een grafiek van het geluidssignaal gegenereerd. Mustafa zal zich vandaag bezig houden met de FFT analyse.
\subsection*{Dinsdag}
FFT wordt weergegeven in een grafiek, lijkt goed te lopen. Bluetooth werkt nogsteeds niet goed. Jos en Mustafa gaan hier samen aan werken, beginnend bij het begin.
\subsection*{Woensdag}
Bluetooth connectie met twee andere devices is nu goed mogelijk. Er is gebleken dat meer connecties in praktijk niet mogelijk zijn. Jos en Mustafa beginnen met het mergen van de code. Met Ilya bespreken of Wifi wellicht een beter alternatief is.
\subsection*{Vrijdag}
Jos en Mustafa waren erg tevreden met hun geboekte vooruitgang op woensdag. Bluetooth is operationeel en de streaming recorder is zondanig aangepast dat het mogelijk is om al streamende FFT analyses uit te voeren. Er is besloten om een sonar systeem te implementeren. Indien een appraat een geluidje hoort, zal hij antwoorden met een nieuw geluidje. 

\section*{Week 3}
\subsection*{Maandag}
Piepherkenning is grotendeels af, Jos en Mustafa gaan zich richten op de foutjes in de netwerkcode. Jos zal daarna de protocollen verder uitwerken. Mustafa zal zich daarna bezig houden met het timen van piepjes.
\subsection*{Dinsdag}
Netwerk code verder testen en gladstrijken en de protocollen voor het afspelen van en luisteren naar piepjes. Metingen doen om afstand en nauwkeurigheid tussen twee mobieltjes te bepalen.

\end{document}
