\documentclass[12pt]{article}

\usepackage[dutch]{babel}
\usepackage{a4wide}
\usepackage{amsmath}
\usepackage{enumerate}
\usepackage{eurosym}
\usepackage{url}
\usepackage{graphicx}

\author{Jos Bonsink \& Mustafa Karaalioglu}

\begin{document}

\title{Backlog - Project Thunder}
\maketitle

\section*{Inleiding}
Het werk kan worden opgedeeld in twee categorie\"en, audio, bluetooth en lokalisatie.

\section*{Audio}
\begin{enumerate}
\item Geluid afspelen
\item Geluid opnemen
\item Geluid analyseren en beep localiseren
\end{enumerate}

\section*{Bluetooth}
\begin{enumerate}
\item Bluetooth apparaten in de omgeving kunnen vinden
\item Verbinden met andere bluetooth apparaten
\item Berichten over en weer kunnen sturen
\end{enumerate}

\section*{Lokalisatie}
\begin{enumerate}
\item Bepaal met behulp van geluid de afstanden tussen de verschillende telefoons.
\item Als er drie afstanden bekend zijn, kan de relatieve positie t.o.v. van de andere drie worden berekend.
\end{enumerate}

\end{document}



